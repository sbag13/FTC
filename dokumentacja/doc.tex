\documentclass[12pt, notitlepage]{article}

\usepackage{geometry}
\usepackage{polski}
\usepackage[utf8]{inputenc}
\usepackage[T1]{fontenc}
\usepackage{enumitem}
\usepackage{graphicx}
\usepackage{float}
\usepackage{listings}
\usepackage{url}
\lstset{basicstyle=\footnotesize\ttfamily,breaklines=true}

\usepackage{etoolbox}
\makeatletter
\patchcmd{\chapter}{\if@openright\cleardoublepage\else\clearpage\fi}{}{}{}
\makeatother

\usepackage[toc]{appendix}
\renewcommand{\appendixtocname}{Dodatki}
\renewcommand\refname{Odwołania}
\usepackage[parfill]{parskip}
% \setlength{\parindent}{0pt}
% \setlength{\parskip}{\baselineskip}

% \usepackage[
%     backend=biber,
%     style=alphabetic,
%     sorting=ynt
%     ]{biblatex}

% \addbibresource{doc.bib}
\geometry{legalpaper, margin=0.8in}

\begin{document}

\begin{titlepage}
    \thispagestyle{empty}
    \title{\textbf{\Huge Systemy bezpieczne i FTC \\[1cm]\LARGE Wiarygodność systemów w pracy zespołowej - \\ jak czynnik ludzki wpływa na różnice w implementacji specyfikacji}}
    \author{Szymon Bagiński \\ Artur Walasz \\[1cm]{\small Prowadzący: Mgr inż. Tomasz Serafin}}
    % \author{Szymon Bagiński\thanks{funded by the ShareLaTeX team}}
    \date{Styczeń 2018}
    \maketitle
    \vfill
    % \renewcommand{\chapter}[2]{}
    \begin{center}
        \Large \bfseries\contentsname
    \end{center}
    \tableofcontents
    \vfill
\end{titlepage}    

% \chapter{Wstęp}
% \addcontentsline{toc}{chapter}{Wstęp}

\section{Wstęp}

\subsection{Cel projektu}

Wykorzystywanie nawet najlepszego oprogramowania, w którym zaimplementowano najbardziej zaawansowane technologie i najbezpieczniejsze algorytmy, nie jest w stanie zapewnić systemowi 100-procentowego bezpieczeństwa. Dzieje się tak dlatego, że w rozwoju i implementacji oprogramowania uczestniczą ludzie, którzy z natury mają skłonność do popełniania błędów. W rezultacie, ludzie, którzy są częścią systemu, zawsze będą najsłabszym ogniwem bezpieczeństwa systemu. Czynnik ludzki jest głównym powodem, dla którego udaje się tak wiele ataków na komputery i systemy.

Celem tego projektu było zbadanie wpływu czynnika ludzkiego na różne implementacje tej samej specyfikacji. W jego ramach stworzono dwie niezależne implementacje oprogramowania z jednolitym, ustandaryzowanym interfejsemm, sposobem testowania oraz wymaganiami funkcjonalnymi. Na ich podstawie wyciągnięto wnioski na temat tego, jak indywidualne podejście, interpretacja dostarczonych wymagań oraz wybór technologii wpływają na finalną wersję wytworzonego produktu.

\subsection{Realizacja}

Punktem wyjściowym do projektu było stworzenie wspólnej specyfikacji wymagań projektowych oraz zbioru testów akceptacyjnych \cite{tests} prostego serwisu aukcyjnego w architekturze klient-serwer.

Na potrzeby projektu zdefiniowano specyfikację interfejsu serwisu uwzględniającą następujące funkcje:
\begin{itemize}
    \item Rejestracja użytkowników
	\item Logowanie użytkowników
	\item Tworzenie ofert typu:
    \begin{itemize}
		\item akucja: ``auction''
		\item teraz: ``buynow''
    \end{itemize}
	\item Kupowanie / licytacja twoarów (ofert) przez użytkowników
    \item Przeglądanie ofert z filtracją
    \begin{itemize}
        \item opis zawiera tekst zadany tekst
        \item cena minimalna
        \item cena maksymalna
        \item ilość dostępnych przedmiotów
    \end{itemize}
	\item Przeglądanie własnych ofert
    \item Obsługa własnych ofert:
    \begin{itemize}
		\item modyfikacja ofert
        \item usówanie ofert
    \end{itemize}
\end{itemize}

Specyfikacja zawiera opis wszystkich wymaganych endpointów (dodatek \ref{app:endpoints}), w tym:
\begin{itemize}
    \item adres url
    \item dozwolone metody HTTP
    \item wymagany format danych, które powinny zostać przesłane w zapytaniach POST
    \item oczekiwane statusy odpowiedzi w zależności od warunków
\end{itemize}

Dokładne wymagania dla konkretnych funkcji serwizu zostaną opisane w dalszej części dokumentacji. Porównane zostaną obydwie implementacjie pod kątem zgodności z dokumentacjią oraz ze sobą wzajemnie.

\section{Porównanie implementacji}

\subsection{Wybór technologii}

Podstawową różnicą między dwiema powstałymi implementacjami jest użyta technologia. Każde z rozwiązań posiada oczywiście inny zestaw cech, niektóre rzeczy wykonuje się łatwiej kosztem innych. Każda z nich ma także inne ograniczenia, czy zachowania domyślne. Wybór ten zatem mocno rzutuje na różnice, które będą widoczne przy konkretnych funkcjach systemu, bądź na nakład pracy jaki trzeba włożyć przy spełnianiu konkretnych wymagań.

\subsubsection{Implementacja 1}

Do pierwszej implementacji postanowiono użyć języka programowania Rust \cite{rust}. Jest to kompilowalny język stworzony z myślą o tworzeniu stabilnych, bezpiecznych, i przede wszystkim wydajnych aplikacji.

Do stworzenia aplikacji, korzystającej z protokołu HTTP, skorzystano z biblioteki Rocket \cite{rocket}. Jest to bardzo wygodne rozwiązanie. W połączeniu z menedżerem pakietów Cargo, pierwszą najprostszą aplikację można stworzyć w kilka minut. Rozszerzanie jej o kolejne funkcjonalności również nie przyspaża żadnych problemów. Jednocześnie zapewnia ona bezpieczeństwo danej aplikacji, elastyczność a przedwszystkim stabilność (m. in. poprzez ``type safety''). W połączeniu z ogólnymi cechami języka Rust daje nam to duże prawdopodobieństwo, że skompilowany program, będzie działał bez problemów przez długi czas.

Kod źródłowy można znaleźć pod poniższym linkiem:\\
\centerline{\url{https://github.com/sbag13/FTC/tree/master/baginski}}

\subsubsection{Implementacja 2}

<TODO> Artur, jak leci? </TODO>

\subsection{Rejestracja} \label{sec:rejestracja}

Mogłoby się wydawać, że tak prosty \textsl{endpoint} jak rejestracja nie powinien spowodować dużych rozbieżności. Niemniej jednak niedociągnięcia specyfikacji pozostawiają pole do różnych interpretacji.\\
W ciele zapytania POST podczas rejestracji wysyłana jest struktura JSON z dwoma polami: mail oraz password. Wartość pola mail musi być oczywiście niepowtarzalna. Istniej zatem możliwość aby użyć jej jako identfikatora danego użytkownika, zarówno w bazie danych jak i w aplikacji. Tak postanowiono zrobić w implementacji 1., gdzie mail jest kluczem pierwotnym dla użytkownika. <TODO> Artur, jak z tym mailem u cb? </TODO> \\
Następnie należy rozpatrzyć przypadek, którego specyfikacja nie przewiduje. Gdy nie uda się poprawnie zinterpretować przesłanych danych jako JSON, bibliotek Rocket obsłuży taki przypadek wysyłając status 422 (Unprocessable Entity). <TODO> Artur, jak z tym u Cb, dodaj wniosek jak możesz </TODO> \\
Specyfikacja określa co powinno się stać gdy zapytanie zostanie wysłane poprawnie za wyjątkiem typu metody HTTP. W takim przypadku powinien zostać zwrócony status 405 (Method Not Allowed). W przypadku implementacji 1. skutkuje to bardzo dużym nakładem mało istotnej pracy. Rocket, gdy nie uda mu się dopasować żadnego istniejącego endpointu, wysyła domyślnie status 404 (Not Found). Dzieje się tak również w przypadku gdy nie znajdzie odpowiedniej metody HTTP, nawet jeśli reszta parametrów się zgadza. Aby otrzymać zgodność ze specyfikacją należałoby więc utworzyć endpointy dla wszystkich możliwych metod protokołu HTTP, których jedyną odpowiedzialnością byłoby wysłanie błędu o statusie 405. W implementacji 1. zdecydowano się nadpisać tylko metody get, put oraz delete. <TODO> Artur, a jak z tym u Cb? </TODO>

\subsection{Logowanie}

Przy implementacji logowania napotkano problemy opisane w sekcji \ref{sec:rejestracja}. Podczas tworzenia specyfikacji postanowiono, że do uwierzytelniania będą wykorzystywane tokeny JWT (JSON Web Token) \cite{jwt}. Jest to jednak elastyczna struktura i jej elementy nie zostały jednoznacznie określone. W przypadku implementacji 1. nagłówek JWT został wygenerowany w sposób domyślny, który używa algorytmu HS256, a ładunkiem tokena zostało tylko pole mail danego użytkownika. W żaden sposób nie został określony czas ważności sesji. <TODO> Artur, jak u Cb z tymi tokenami? też wrzuciłeś JWT do ciasteczek?</TODO>

Rażącym niedopatrzeniem podczas tworzenia dokumentacji jest także fakt, że nie pomyślano o funkcji wylogowwywania. Aby usunąć sesję należy więc zalogować się na konto innego użytkownika albo usunąć pliki ``cookies'' z przeglądarki. 

\subsection{Tworzenie ofert}
\subsection{Kupowanie / licytacja}
\subsection{Przeglądanie ofert wraz z filtrowaniem}

\section{Podsumowanie}

\begin{thebibliography}{9}

\bibitem{jwt}
\textit{JSON Web Tokens} [online], Data dostępu: 12.01.2019. 
\newline\url{https://jwt.io/}

\bibitem{tests}
\textit{Opis testów funkcji systemu} [online], Data dostępu: 12.01.2019. 
\newline\url{https://drive.google.com/drive/folders/194qvJLmUYSD427WfbAkm365Q3SCY1ZaY}.

\bibitem{rocket}
\textit{Rocket - Simple, Fast, Type-Safe Web Framework for Rust} [online], Data dostępu: 12.01.2019. 
\newline\url{https://rocket.rs/}.

\bibitem{rust}
\textit{Rust} [online], Data dostępu: 12.01.2019. 
\newline\url{https://www.rust-lang.org/}.

\end{thebibliography} 


\newpage
\setlength\parindent{24pt}
\begin{appendices}
\section{Interfejs serwisu aukcyjnego} \label{app:endpoints}
\footnotesize
\texttt{
/registration \\
POST \\
\{\\
\indent mail: \indent		string,\\
\indent	password:	string\\
\}\\
\\
201 - Created (pomyślne utworzenie użytkownika)\\
400 - Bad Request (np. zły email)\\
409 - Conflict (konto istnieje)\\
500 - Internal Server Error\\
503 - Server Unavaiable (powód podany w opisie)\\
\\
not POST \\
405 - Method Not Allowed 
\\ ---------------------------------------------------------------------------------------------------------------------------------
\\
/login\\
POST \\
\{\\
\indent    mail:	\indent	string,\\
\indent    password:	string\\
\}\\
\\
200 - OK, (JWT token in response)\\
401 - Unauthorized\\
404 - Not Found\\
500 - Internal Server Error\\
503 - Server Unavaiable (powód podany w opisie)\\
\\
not POST\\
405 - Method Not Allowed\\
---------------------------------------------------------------------------------------------------------------------------------\\
/offers\\
POST\\
\{\\
\indent type*: \ \ \ \ \ \ \ \lbrack auction|buynow\rbrack,\\
\indent	description*:	string,\\
\indent	price*:	\ \ \ \ \ \ float, (cena min. - akcja; cena za sztukę - kup teraz),\\
\indent	date*:\ \ \ \ \ \ \ \ 	timestamp, (sekundy, tylko dla aukcji),\\
\indent	amount*: \ \ \ \ \ int (tylko dla kup teraz)\\
\}\\
\\
201 - added \\
	\{\\
    	\indent offer\_id: int\\
	\}\\
400 - Bad Request (niepoprawny JSON, brakujące lub nadmiarowe pola)\\
401 - Unauthorized (niezalogowany użytkownik)\\
403 - Forbidden (zalogowany na nieuprawnione konto)\\
500 - Internal Server Error\\
503 - Server Unavaiable (powód podany w opisie)\\
\\
GET\\
Dozwolone filtry w url:\\
- contains - pole description zawiera ciąg znaków\\
- price\_min - minimalna cena towaru (i aukcji)\\
- price\_max - maksymalna cena towaru (i aukcji)\\
- type - [auction/buynow] - typ oferty\\
- created\_by\_me: boolean\\
\\
200 - (oferty w odpowiedzi - może być pusta "\lbrack\rbrack")\\
400 - Bad Request (niepoprawny filtr w url)\\
500 - Internal Server Error\\
503 - Server Unavaiable (powód podany w opisie)\\
---------------------------------------------------------------------------------------------------------------------------------\\
/offers/\{id\}\\
PATCH\\
\{\\
\indent    **fields\_to\_modify...\\
\}\\
\\
użytkownik może modyfikować:\\
- price, amount, description - dla ofert typu "buynow"\\
- price, description -  dla ofert typu "auction"\\
\\
202 - Accpepted\\
400 - Bad Request\\
403 - Unauthorized\\
404 - Not Found (nie znaleziono ofert)\\
\\
DELETE\\
(no payload)\\
202 - Accepted\\
403 - Unauthorized\\
404 - Not Found (nie znaleziono oferty)\\
\\
GET\\
(no payload)\\
200 - Ok (w odpowiedzi szczegóły aukcji)\\
\{	\\
\indent    type: \ \ \ \ \ \ \ \ \ "auction",\\
\indent    description: \ \ "opis",\\
\indent    status: \ \ \ \ \ \ \ [active / expired],\\
\indent    last\_bid: \ \ \ \ \ float,\\
\indent    customer\_id: \ \ int,\\
\indent    expiration\_ts: <timestamp>\\
\}\\
200 - Ok (szczegóły oferty "buynow")\\
\{	\\
\indent type: \ \ \ \ \ \ \ "buynow",\\
\indent description: string,\\
\indent	price: \ \ \ \ \ \ int,        \\
\indent	amount: \ \ \ \ \ int\\
\}\\
404 - Not Found\\
---------------------------------------------------------------------------------------------------------------------------------\\
/offers/\{id\}/buy\\
POST\\
\{\\
\indent    bid: \ \ \  int (auction)\\
\indent    amount:	int (buynow)\\
\}\\
\\
202 - Accepted\\
400 - Bad Request (niepoprawny JSON, brakujące lub nadmiarowe pola)\\
401 - Unauthorized (niezalogowany użytkownik)\\
409 - Conflict:\\
\indent	409 - \{"minimal\_bid": <minimalny bid>\}\\
\indent	409 - \{"max\_amout": <ilość dostępnych produktóœ>\}\\
\indent	409 - \{"status": "expired"\}\\
\indent	409 - \{"conflict": "unable to order own items"\}\\
\indent	409 - \{"conflict": "you can not bid on the auction you win"\}\\
500 - internal server error\\
503 - Server Unavaiable (powód podany w opisie)\\
}

\end{appendices}

\end{document}