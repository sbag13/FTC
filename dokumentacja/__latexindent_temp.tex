\documentclass[12pt, notitlepage]{report}

\usepackage{geometry}
\usepackage{polski}
\usepackage[utf8]{inputenc}
\usepackage[T1]{fontenc}
\usepackage{enumitem}
\usepackage{graphicx}
\usepackage{float}
\usepackage{listings}
\lstset{basicstyle=\footnotesize\ttfamily,breaklines=true}

\usepackage{etoolbox}
\makeatletter
\patchcmd{\chapter}{\if@openright\cleardoublepage\else\clearpage\fi}{}{}{}
\makeatother

\usepackage[toc]{appendix}
\renewcommand{\appendixtocname}{Dodatki}

\usepackage[
    backend=biber,
    style=alphabetic,
    sorting=ynt
    ]{biblatex}

\addbibresource{doc.bib}
\geometry{legalpaper, margin=0.8in}

\begin{document}

\begin{titlepage}
    \thispagestyle{empty}
    \title{\textbf{\Huge FTC czy tam FCT \\[1cm]\LARGE dwie implementacje}}
    \author{Szymon Bagiński \\ Artur Walasz \\[1cm]{\small Prowadzący: nie pamiętam chwilowo}}
    % \author{Szymon Bagiński\thanks{funded by the ShareLaTeX team}}
    \date{Styczeń 2018}
    \maketitle
    \vfill
    \renewcommand{\chapter}[2]{}
    \begin{center}
        \Large \bfseries\contentsname
    \end{center}
    \tableofcontents
    \vfill
\end{titlepage}    

\chapter*{Wstęp}
\addcontentsline{toc}{chapter}{Wstęp}
    \textit{EEG} \\
    \textbf{Python} \\
    \textsl{cos tam Cos tam} \\
    \texttt{code code code} \\

    \begin{itemize}
        \item jeden
        \item drugi
        \item trzeci
    \end{itemize}    
    
    \url{https://github.com/sbag13/EEG_sleep_stages_recognition}.

\chapter{jakis tam rozdzial}
\Huge Huge
\LARGE LARGE
\Large Large
\normalsize normalsize
\small small
\tiny tiny
\\
\normalsize nie pamiętam jakie tam jeszcze były, można sprwadzic

\begin{lstlisting}[caption=Interfejs klasy RubyInterpreter, label=rubyInterfejs]
class RubyInterpreter
{
    public:		
        static RubyInterpreter * getInterpreter () ;
        ~RubyInterpreter () ;
        void runScript (const char * script) ;
        template<class VariableType >
        VariableType getRubyVariable (const std::string & variableName) ;
        ModificationRhoU modifyNodeLayout 
            (NodeLayout & nodeLayout, const std::string & rubyCode) ;

    private:
        RubyInterpreter () ;
        void initializeRubyInterpreter () ;
        static RubyInterpreter * instance_ ;
};
\end{lstlisting}

\newpage
nowa strona

    \section{sekcja}
        trzeba\_escapowac\_rozne\_znaki

        Wyliczenia numerowane
        \begin{enumerate}
            \item łan
            \item tu
            \item tri           
        \end{enumerate}

        \subsection{podsekcja}

\chapter{nastepny rozdzial}
    odniesienie do jakiejs labelki \ref{fig:rust}.
        
    \begin{figure}[H] % H - obraz nie pływa po dokumencie, inline tak jakby
        \centering
        \includegraphics[width=0.3\textwidth]{rust.png}
        \caption{Rust rules}
        \label{fig:rust}
    \end{figure}

    cos tam
\newline
nowa linie \\
nastepna linia
% komentarz

hardy wzorek: $$-\frac{1}{n}\sum_{j=1}^{n}y_j^{(i)}\log(y_{j\_}^{(i)}) + (1  - y_j^{(i)})\log(1 - y_{j\_}^{(i)})$$

\chapter*{Podsumowanie}
\addcontentsline{toc}{chapter}{Podsumowanie}

takze tego

Bibliografię też się da ale zgłowy nie pamiętam, ogarnie się

% \printbibliography[heading=bibintoc, title={Bibliografia}]

\begin{appendices}
    \chapter{Pliki wykorzystane do treningu}
    \label{appendix:A}
    SC4001E0-PSG.edf \newline
    SC4002E0-PSG.edf \newline
    SC4011E0-PSG.edf \newline
    SC4012E0-PSG.edf \newline
    SC4021E0-PSG.edf \newline
    SC4022E0-PSG.edf \newline
    SC4031E0-PSG.edf \newline
    SC4032E0-PSG.edf \newline
    SC4041E0-PSG.edf \newline
    SC4042E0-PSG.edf \newline
    SC4051E0-PSG.edf \newline
    SC4052E0-PSG.edf \newline
    SC4061E0-PSG.edf \newline
    SC4062E0-PSG.edf \newline
    SC4071E0-PSG.edf \newline
    SC4072E0-PSG.edf \newline
    SC4081E0-PSG.edf \newline
    SC4082E0-PSG.edf \newline
    SC4091E0-PSG.edf \newline
    SC4092E0-PSG.edf \newline
    SC4122E0-PSG.edf \newline
    SC4131E0-PSG.edf \newline
    SC4141E0-PSG.edf \newline
    SC4142E0-PSG.edf \newline
    SC4151E0-PSG.edf \newline
    SC4152E0-PSG.edf \newline
    SC4161E0-PSG.edf

    \chapter{Pliki wykorzystane do testowania}
    \label{appendix:B}
    SC4101E0-PSG.edf \newline
    SC4102E0-PSG.edf \newline
    SC4111E0-PSG.edf \newline
    SC4112E0-PSG.edf \newline
    SC4121E0-PSG.edf

\end{appendices}

\end{document}